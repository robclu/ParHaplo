%%%%%%%%%%%%%%%%%%%%%%%%%%%%%%%%%%%%%%%%%%%%%%%%%%%%%%%%%%%%%%%%%%%%%%%%%%%%%%%%%%%%%%%%%%%%%%%%%%%%%%%%%%%%%
%
% witseiepaper-2005.tex
%
%                       Ken Nixon (12 October 2005)
%
%                       Sample Paper for ELEN417/455 2005
%
%%%%%%%%%%%%%%%%%%%%%%%%%%%%%%%%%%%%%%%%%%%%%%%%%%%%%%%%%%%%%%%%%%%%%%%%%%%%%%%%%%%%%%%%%%%%%%%%%%%%%%%%%%%%%

\documentclass[10pt,twocolumn]{witseiepaper}

%
% All KJN's macros and goodies (some shameless borrowing from SPL)
\usepackage{KJN}
\usepackage{textcomp}
%
% PDF Info
%
\ifpdf
\pdfinfo{
/Title (Parallel Individual Haplotyping Assembly : Xeon Phi vs. Nvidia K20x )
/Author (Robert Clucas)
/CreationDate (D:200309251200)
/ModDate (D:200510121530)
/Subject (ELEN4002 Laboratory Project, 2015)
/Keywords (Haplotyping, GPU, Branch, Bound, Simplex)
}
\fi

%%%%%%%%%%%%%%%%%%%%%%%%%%%%%%%%%%%%%%%%%%%%%%%%%%%%%%%%%%%%%%%%%%%%%%%%%%%%%%%%%%%%%%%%%%%%%%%%%%%%%%%%%%%%%

\begin{document}

\title{Parallel Individual Haplotying Assembly : Xeon Phi vs. Nvidia K20x}

\author{Robert J. clucas
\thanks{School of Electrical \& Information Engineering, University of the
Witwatersrand, Private Bag 3, 2050, Johannesburg, South Africa}
}

%%%%%%%%%%%%%%%%%%%%%%%%%%%%%%%%%%%%%%%%%%%%%%%%%%%%%%%%%%%%%%%%%%%%%%%%%%%%%%%%%%%%%%%%%%%%%%%%%%%%%%%%%%%%%

\abstract{}

\keywords{Brach, Bound, GPU, Haplotyping, Simplex}

\maketitle
\thispagestyle{empty}\pagestyle{empty}


%%%%%%%%%%%%%%%%%%%%%%%%%%%%%%%%%%%%%%%%%%%%%%%%%%%%%%%%%%%%%%%%%%%%%%%%%%%%%%%%%%%%%%%%%%%%%%%%%%%%%%%%%%%%%

\section{INTRODUCTION}

It is commonly accepted that all humans share $\mathtt{\sim}$99$\%$ of the same DNA, however, the small variations 
cause the human beings to have different physical traits. Single nucleotide polymorphisms (SNPs), which is
the variation of a single DNA base from one individual to another, and are believed to be able to address
genetic differences. For diploid organisms, which have pairs of chromosomes, a \textit{haplotype} is a 
sequence of SNPs in each copy of a pair of chromosomes. A \textit{genoype} describes the conflated data of the
haplotypes on a pair of chromosomes. Haplotypes are believed to contain more generic information than
genotypes \cite{stephens:2001}.Obtaining haplotypes correctly is a difficult problem, which is broken into 
two subdomains: individual haplotype assembly and haplotype inference. \\
Haplotype inference uses the genotype of a set of individuals. The genotype data tells the status of each
allele at a position, but does not distinguish which copy of the chromosome the allele came from. CITE (HE)
This negative aspects of this approach are that it cannot distinguish rare and novel SNPs, and there is no way
of knowing if the inferred haplotype is completely correct. \\
Individual haplotype assembly uses fragments of sequences generated by sequencing technology to determine
haplotypes. The fragments of a sequence come from the two copies of an individual's chromosome, the goal of the
individual haplotyping problem is to correctly determine two haplotypes, where each haplotype corresponds to
one of the two copies of the chromosome.

%%%%%%%%%%%%%%%%%%%%%%%%%%%%%%%%%%%%%%%%%%%%%%%%%%%%%%%%%%%%%%%%%%%%%%%%%%%%%%%%%%%%%%%%%%%%%%%%%%%%%%%%%%%%%

\section{BACKGROUND}

%%%%%%%%%%%%%%%%%%%%%%%%%%%%%%%%%%%%%%%%%%%%%%%%%%%%%%%%%%%%%%%%%%%%%%%%%%%%%%%%%%%%%%%%%%%%%%%%%%%%%%%%%%%%%


%%%%%%%%%%%%%%%%%%%%%%%%%%%%%%%%%%%%%%%%%%%%%%%%%%%%%%%%%%%%%%%%%%%%%%%%%%%%%%%%%%%%%%%%%%%%%%%%%%%%%%%%%%%%%

%\nocite{*}
\bibliographystyle{witseie}
\bibliography{sample}

%{\tiny \vfill \hfill \today \hspace{5mm} witseie-paper-2003.\TeX}

\end{document}

